%-----------------------------------------------------------------------------
% Beginning of journal.tex
%------------------------------------------------------------------------------
%
% AMS-LaTeX version 2 sample file for journals, based on amsart.cls.
%
%        ***     DO NOT USE THIS FILE AS A STARTER.      ***
%        ***  USE THE JOURNAL-SPECIFIC *.TEMPLATE FILE.  ***
%
% Replace amsart by the documentclass for the target journal, e.g., tran-l.



\documentclass[12pt]{amsart}
\usepackage{amsmath, amsthm, amssymb, tikz-cd}
\usepackage{xspace}
\usepackage{rotating,mathtools}
\usepackage[all,cmtip]{xy}\xyoption{dvips}
\usepackage[hidelinks]{hyperref}
\usepackage{listings}
\usepackage{xcolor}
\lstset { %
    language=C++,
    backgroundcolor=\color{black!5}, % set backgroundcolor
    %basicstyle=\footnotesize,% basic font setting
    basicstyle=\ttfamily\color{black},
  moredelim=**[is][\color{gray}]{@}{@},
}


\newtheorem{theorem}{Theorem}[section]
\newtheorem{lemma}[theorem]{Lemma}
\newtheorem{corollary}[theorem]{Corollary}
\newtheorem{prop}[theorem]{Proposition}

\theoremstyle{definition}
\newtheorem{definition}[theorem]{Definition}
\newtheorem{example}[theorem]{Example}
\newtheorem{xca}[theorem]{Exercise}

\theoremstyle{remark}
\newtheorem{claim}[theorem]{Claim}
\newtheorem{fact}[theorem]{Fact}
\newtheorem{remark}[theorem]{Remark}
\newtheorem{question}[theorem]{Question}
\newtheorem{akn}[theorem]{Acknowledgments}
\newtheorem{assumption}[theorem]{Assumption}

%%%%%%%%%%%%%%%%%%%%%%%%%%%%%%%%%%%%%%%%%%%
%
%              Commutative algebra
%
%%%%%%%%%%%%%%%%%%%%%%%%%%%%%%%%%%%%%%%%%%%
\newcommand{\supp}[1]{\textrm{supp}(#1)}
\newcommand{\depth}[2]{\textrm{depth}_{#1}({#2})}
\newcommand{\height}[1]{\textrm{ht}{#1}}
\newcommand{\grade}[1]{\textrm{grade}(#1)}
\newcommand{\spec}[1]{\textrm{spec}#1}
%


%%%%%%%%%%%%%%%%%%%%%%%%%%%%%%%%%%%%%%%%%%%
%
%             Sheaf
%
%%%%%%%%%%%%%%%%%%%%%%%%%%%%%%%%%%%%%%%%%%%

\newcommand{\sshf}[1]{\mathcal{O}_{#1}}
\newcommand{\shf}[1]{\mathcal{#1}}
\newcommand{\proj}[1]{\textrm{proj}(#1)}
\newcommand{\prj}[1]{\mathbb{P}^{#1}}
\newcommand{\E}{\mathbb{E}}
\newcommand{\R}{\mathcal{R}}
\newcommand{\rH}{\mathcal{H}}
\newcommand{\F}{\mathcal{F}}
%%%%%%%%%%%%%%%%%%%%%%%%%%%%%%%%%%%%%%%%%%%
%
%             Operations
%
%%%%%%%%%%%%%%%%%%%%%%%%%%%%%%%%%%%%%%%%%%%
\newcommand{\ts}[1]{\otimes_{#1}}
\newcommand{\iso}{\simeq}

\newcommand{\ch}[1]{\textrm{ch}(#1)} %Character function%
\newcommand{\pf}[1]{\textrm{pf}(#1)}
\newcommand{\rk}[1]{\textrm{rank}(#1)}
\newcommand{\is}[1]{\mathscr{I}_{#1}}




%%%%%%%%%%%%%%%%%%%%%%%%%%%%%%%%%%%%%%%%%%%
%
%                      Tilde
%
%%%%%%%%%%%%%%%%%%%%%%%%%%%%%%%%%%%%%%%%%%%


\newcommand{\tX}{\widetilde{X}}
\newcommand{\tA}{\widetilde{A}}
\newcommand{\tE}{\widetilde{E}}
\newcommand{\bX}{\overline{X}}


%%%%%%%%%%%%%%%%%%%%%%%%%%%%%%%%%%%%%%%%%%%
%
%                     Arrows and Sequencies
%
%%%%%%%%%%%%%%%%%%%%%%%%%%%%%%%%%%%%%%%%%%%

\newcommand{\ra}{\rightarrow}
\newcommand{\M}{\mathcal{M}}
\newcommand{\lra}{\longrightarrow}
\newcommand{\embedding}{\hookrightarrow}
\newcommand{\map}[3]{#1: #2\rightarrow#3}
\newcommand{\ses}[3]{0\rightarrow#1\rightarrow#2\rightarrow#3\rightarrow{0}}

%%%%%%%%%%%%%%%%%%%%%%%%%%%%%%%%%%%%%%%%%%%
%
%                     Dualizing complex
%
%%%%%%%%%%%%%%%%%%%%%%%%%%%%%%%%%%%%%%%%%%%
\newcommand{\oX}{\omega _X^{\bullet}}
\usepackage{enumerate}
\usepackage[hidelinks]{hyperref}
\usepackage{xcolor}
\hypersetup{
    colorlinks,
    linkcolor={blue!50!black},
    citecolor={green!50!black},
    urlcolor={red!80!black}
}

\textwidth= 6.5in
\textheight=8.75in
\voffset-.5in
\hoffset-.75in
\marginparwidth=56pt



\begin{document}
\allowdisplaybreaks
\tableofcontents
\title{}

%    Information for first author
\author{}


%    Address of record for the research reported here

\dedicatory{}

\keywords{}

\maketitle

\section{Simulation of Random numbers}
\subsection{Poisson Distribution}

We use inverse function method.
First note that Poisson distribution satisfies 
\begin{equation*}
P(N= k+1) = P(N= k)\cdot \frac{\lambda}{k+1}
\end{equation*}
Then for $U\sim U(0,1)$, 
\begin{equation*}
F(k) = P_U(U\le F(k)))
\end{equation*}
So we first generate $U$, then find the first $K$ such that $U\le F(k)$.

The pseudo code is the following

\begin{lstlisting}[escapeinside={(*}{*)}]
p = exp((*$- \lambda$ *))
F = p
k = 0
@/@@/@@  generate u from U(0,1) @  
while U > F:
	k += 1
	p *= (* $\frac{\lambda}{k}$ *)
	F += p
return k	


\end{lstlisting}



\section{BSM model}

We assume the stock price satisfies the following SDE, 
\begin{equation*}
dS =  S \mu dt + S\sigma dw,
\end{equation*}
where $w$ is standard brownian motion.
By Ito's lemma, we have 
\begin{equation*}
S(t) = S(0) e^{(\mu -\frac{1}{2}\sigma ^2)t + \sigma w(t)}.
\end{equation*}
So to simulate the discounted stock price $e^{-rt}S(t)$ we need
 \begin{itemize}
 \item Current stock price $S(0): $ spot
 \item expiration date: T

 \item Interest rate $r$: r
  \item Average return $\mu$: r. Because using risk neutral pricing, we can show $\mu = r$.
 \item Volatility $\sigma$: vol
 \item The Brownian motion $w(t)$
 
 \end{itemize}

\subsection{Estimation of $w(t)$}

To create Brownian motion, we begin with symmetric random walk.
Let 
\begin{equation*}
X_k = \begin{cases} +1, & \mbox{if } \mbox{ Head} \\ -1, & \mbox{if } \mbox{Tail} \end{cases}
\end{equation*}
and
\begin{equation*}
M_t = \sum _{k = 1}^{t} X_k
\end{equation*}
To approximate Brownian motion, we define the {\it scaled symmetric random walk} by
\begin{equation*}
W^{(n)}(t)= \frac{1}{\sqrt{n}} M_{nt}
\end{equation*}
\begin{itemize}
\item We move more frequently, from $t$ times to $nt$ times in the same period of time $t$.
\item $E[W^{(n)}(t)]= 0$
\item Var $W^{(n)}(t)$ =  t
\item By central limit theorem
\begin{equation*}
W^{(n)}(t) =  \frac{1}{\sqrt{n}} M_{nt} = \frac{\sum X_k}{\sqrt{n}} = \sqrt{t}\frac{\sum X_k}{\sqrt{n} \sqrt{t}} \Rightarrow N(0,t)
\end{equation*}
\end{itemize}

\section{Jump-Diffusion  model}
\subsection{General Setup}
Assuming there are some jumps of the stock price, then the Jump-Diffusion model is 
\begin{equation*}
\frac{dS(t)}{S(t-)} = \mu dt +\sigma dW+dJ(t),
\end{equation*}
where 
\begin{equation*}
S(t-) = \lim _{\tau \rightarrow t^-}S(\tau)
\end{equation*}
and   the jump process.
\begin{equation*}
J(t) = \sum_{j=1}{\tau_N \le t}(Y_j-1)
\end{equation*}
We assume the jumps will happen at 
\begin{equation*}
0< \tau_1<\tau_2<\cdots<\tau_N
\end{equation*}
and $Y_j$ are random variables with positive values.

To justify the assumption of $Y_j$ note that
\begin{eqnarray*}
dJ(t) &=& 0 \mbox{  if }  t  \mbox{ is not equal to any of } \tau_j\\
&=& Y_j-1 if t = \tau_j 
\end{eqnarray*}
Then according to the jump diffusion model 
\begin{eqnarray*}
&& S(\tau_j)-S(\tau_j-)=  (Y_j-1)\cdot S(\tau_j-)\\
&\Rightarrow& \frac{S(\tau_j)}{S(\tau_j-)} = Y_j
\end{eqnarray*}
That is, $Y_j$ is the ratio of stock prices before and after the jump.

\begin{prop}
The solution of the jump diffusion model is 
\begin{equation*}
S(t) = S(0)e^{(\mu - \frac{1}{2}\sigma ^2)t +\sigma w  }\prod Y_j
\end{equation*}
\end{prop}

\subsection{Specific Assumptions}
There are two contingent variables involved in the jump diffusion model. The number of jumps,  $N(t)$ and the distributions of the jump ratio, $Y_j$.
To simplify the model, we made the following assumptions,
\begin{itemize}
\item $N(t)$ is a Poisson process with parameter $\lambda$
\item $Y_j$ are independent to $W(t)$ and $N(t)$,  and i.i.d. to log normal distribution $LN(a, b^2)$
\end{itemize}

For the Poisson distribution $N(t)$, the inter arrival times $\tau_{j+1}-\tau_j$ are i.i.d. with exponential distribution
\begin{equation*}
P(\tau_{j+1}-\tau_j \le t)= 1-e^{-\lambda t}
\end{equation*}

Since product of log normal is still log normal, we have 
\begin{equation*}
\prod_{j=1}^{j=n} Y_j \sim LN(na, nb^2)
\end{equation*}

Condition on $N(t) = n$, the stock price
\begin{eqnarray*}
S(t) &=& S(0)\cdot e^{(\mu-\frac{1}{2}\sigma^2)t +\sigma w(t)}\cdot LN(na, nb^2)\\
&=& S(0)\cdot LN((\mu -\frac{1}{2}\sigma ^2)t, \sigma^2 t)\cdot LN(na, nb^2)\\
&=& LN( logS(0)+(\mu - \frac{1}{2}\sigma ^2)t +na, \sigma ^2t+ nb^2)
\end{eqnarray*}

Let $F_{n,t}(x)$ denote the cdf of $S(t)$, then the unconditional distribution of $S(t)\le x$ is 
\begin{equation*}
P(S(t)\le x))= \sum_{n=0}^{\infty} e^{-\lambda t}\frac{(\lambda t)^n}{n!} \cdot F_{n,t}(x)
\end{equation*}

\subsection{Price of options}

We want the discounted value of the stock is a martingale. Look at the diffusion model
\begin{equation*}
\frac{dS(t)}{S(t-)} = \mu dt +\sigma dW+dJ(t),
\end{equation*}
Our first guess is that let $\mu = r$ will work because it will cancel the effect of $r$.
Now we look at $dJ(t)$.
Note that $J(t)$ is a \textbf{compound Poisson distribution}. Use  total law of expectation we see that 
\begin{eqnarray*}
E(J(t)) &=& E(N(t)) E(Y-1)\\
&=& \lambda t E(Y-1)
\end{eqnarray*}
In other words, $J(t)- \lambda t E(Y-1)$ is a martingale. So to make the discounted price of the stock a martingale, we should have
\begin{equation*}
\frac{dS(t)}{S(t-)} = r dt +\sigma dW+dJ(t) - \lambda E(Y-1) dt.
\end{equation*}
So 
\begin{equation*}
\mu = r-\lambda E(Y-1)
\end{equation*}

\subsection{Simulation Of Stock Price}

We consider the method assuming that $t_i's $ are fixed, the number of jumps in each subintervals follows poisson distribution with 
$\lambda$ or $\lambda /T$. Here $T$ is the number of subintervals.
\begin{question}
How do we estimate $\lambda$?
\end{question}


By the solution of the jump diffusion model, we have the following difference equation
\begin{equation*}
S(t_{i+1}) = S(t_i)\cdot e^{(\mu - \frac{1}{2}\sigma ^2)(t_{i+1}-t_i) +\sigma (w(t_{i+1}-w(t_i))  }\prod _{N(t_i)+1}^{N(t_{i+1})}Y_j
\end{equation*}
To make it easier for recursion, we let $x(t)= log (S(t))$ then
\begin{equation*}
x(t_{i+1}) = x(t_{i}) + (\mu - \frac{1}{2}\sigma ^2)(t_{i+1}-t_i) + \sum _{N(t_i)+1}^{N(t_{i+1})} log(Y_j)
\end{equation*}

If we assume further that $Y_j$'s are i.i.d. log normal$(a, b^2)$,  and let $N$ be $N(t_{i+1})-N(t_i)$\\
Then 
\begin{equation*}
\sum _{N(t_i)+1}^{N(t_{i+1})} log(Y_j) = aN + \sqrt{N}b\cdot \mbox{normal}(0,1)
\end{equation*}


In this equation, there are three random numbers to be generated, 
\begin{lstlisting}[escapeinside={(*}{*)}]
Generate Z_1, Z_2 (*$\sim$*) N(0,1)
Generate N (*$\sim$*) Poisson((*$\lambda /T$*))
Generate logY_1+....+logY_N = aN+(*$\sqrt{N}$*)bZ_2 
\end{lstlisting}

\newpage

\section{Interest Rate Model }
We actually want to price the price of bonds like the following equation
\begin{equation*}
P(t, T) = E_t[e^{-\int ^T_tr(u)du}]
\end{equation*}
However there will be too many bonds to price, since different bonds may have different maturity date.
So instead, we model the interest rate first. 
Once we have $r(u)$, then the price of the bonds $P(t,T)$ will automatically be no arbitrage. 
Because the discounted price of $P$ is of the form
\begin{eqnarray*}
e^{-\int ^t_0r(u)du}P(t,T) &=& E_t[ e^{-\int ^t_0r(u)du} e^{-\int ^T_tr(u)du} ]\\
&=& E_t[e^{-\int ^T_0 r(u)du}]\\
&=& E_t[X]
\end{eqnarray*}
where $X$ has nothing to do with $t$. 
So it will be a martingale. We note that if discounted price is martingale then 
there is no arbitrage.


\subsection{Market price of Risk}
Here are some facts about market price of risk.\\
It is a constant $\lambda$ such that 
\begin{equation}\label{price of risk}
\frac{\mu -r}{\sigma} = \lambda
\end{equation}

Choosing a market price of risk is also referred to as choosing a probability measure.
If we have two financial product $f$ and $g$, such that we choose $\sigma _g$ as $\lambda$, then 
$\frac{f}{g}$ is a martingale for all $f$. This $g$ is called numeraire, means the method to discount.
By equation (\ref{price of risk}), we have 
\begin{equation*}
df = (r+\sigma _g \sigma _f)fdt+\sigma_f dw.
\end{equation*}



\subsection{Vasicek Model}
We assume that 
\begin{equation*}
dr = a(b-r)dt + \sigma dz
\end{equation*}
Since interest rate is not directly tradable, we trade derivatives on bounds (?) 
Assume the price of the derivative 
\begin{equation*}
P_t(T) = V(t, T, r)
\end{equation*}
Differentiate it we have 
\begin{equation*}
dV = [ V_t +V_r\cdot a(b-r) +\frac{1}{2}\sigma ^2 V_rr ]dt + V_r \sigma dz
\end{equation*}
Choosing market price of risk equaling 0 (money market?) by equation (\ref{price of risk}) we have 
\begin{equation*}
[ V_t +V_r\cdot a(b-r) +\frac{1}{2}\sigma ^2 V_rr ] = rV
\end{equation*}


\bibliography{bibliography}{}
\bibliographystyle{skalpha}

\end{document}

%------------------------------------------------------------------------------
% End of journal.tex
%------------------------------------------------------------------------------
